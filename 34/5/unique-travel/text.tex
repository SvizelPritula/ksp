\documentclass{article}

\usepackage[a4paper, margin=2cm]{geometry}
\usepackage[utf8]{inputenc}

\usepackage[czech]{babel}
\usepackage{fvextra}
\usepackage{csquotes}
\usepackage{parskip}
\usepackage{expl3}

\usepackage{float}
\usepackage{amsmath}
\usepackage{textcomp}
\usepackage{gensymb}
\usepackage{graphicx}
\graphicspath{{images}}

\usepackage[style=verbose-ibid]{biblatex}
\addbibresource{refs.bib}

\usepackage[hidelinks, unicode, pdfusetitle]{hyperref}

\title{34-5-3 Jednoznačné cestování}
\author{Benjamin Swart}

\setcounter{secnumdepth}{1}

\begin{document}

\maketitle

Pro každé dva vrcholy máme dvě podmínky: zaprvé mezi nimi musí vézt právě jedna nejlevnější cesta a zadruhé musí být nejlevnější cesta vždy i nejkratší.

První podmínku můžeme splnit snadno tak, že se postaráme, aby všechny možné cesty měli různé ceny. Všechny hrany ohodnotíme unikátní mocninou dvojky od \(0\) do \(n-1\). Každá cena od \(0\) do \(2^n-1\) pak unikátně určí kombinaci hran, a tedy i cestu.

Postarali jsme se o to, že žádné dvě ceny nejsou stejně drahé. Teď jen musíme zajistit to, aby nejkratší cesta byla vždy nejlevnější.

To zajistíme tak, že ke všem cenám hran přičteme nějaké velmi velké číslo. Tím dosáhneme toho, že při určování nejkratší cesty bude vždy hrát primární roli její délka a až sekundárně její předchozí cena. Minimální číslo, které musíme přičíst, je o jedna víc než součet předchozích cen všech hran. V našem případě se jedná o \(2^n\).

Pokud tedy máme \(n\) hran očíslovaných \(\left\{0, 1, 2, \dots, n - 1\right\}\), tak tedy stačí každé hraně \(i\) přiřadit cenu \(2^n + 2^i\).

Nejvyšší cena cesty tedy bude řádu \(O(2^n)\). Hrany je triviálně možné ohodnotit v \(O(n)\) čase.

\end{document}
