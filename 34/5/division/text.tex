\documentclass{article}

\usepackage[utf8]{inputenc}
\usepackage[czech]{babel}
\usepackage[a4paper,margin=2cm]{geometry}

\usepackage{parskip}
\usepackage{amsmath}
\usepackage{siunitx}
\usepackage{minted}

\usepackage[unicode]{hyperref}

\setcounter{secnumdepth}{0}

\title{34-5-4 - Dělení ve velkém}
\author{Benjamin Swart}

\begin{document}
\maketitle

\section{Úvod}

Pro přehlednost zdrojový kód z ukázky zabalím do funkce. \texttt{array} je vstupní pole, \texttt{size} je délka tohoto pole a \texttt{max} je horní hranice hodnot v tomto poli (v zadání \(10^6\)).

\begin{minted}{c}
int sum_quotients(int *array, int size, int max) {
    int sum = 0;

    for (int i = 0; i < size; i++) {
        for (int j = 0; j < size; j++) {
            sum += array[i] / array[j];
        }
    }

    return sum;
}
\end{minted}

Snažíme se pro každou dvojici čísel v seznamu vypočítat jejich celočíselný podíl a tyto podíly sečíst.

Celočíselné dělení budu značit operátorem \(\div\).

\section{Jiný způsob počítání}

Dejme tomu, že počítáme celočíselný podíl nějakého čísla \(x\) a 3. Podíl bude 0, pokud \(x \in \left\{0, 1, 2\right\}\), 1, pokud \(x \in \left\{3, 4, 5\right\}\), 2, pokud \(x \in \left\{6, 7, 8\right\}\) a podobně.

Formálně tedy platí, že:

\begin{equation*}
    x \div y = q \equiv x \in \left\{qy, qy+1, qy+2, \dots, qy+y-1\right\}
\end{equation*}

Této vlastnosti můžeme využít. Po každého dělitele \(y\) projdeme všechny možné podíly \(q\) a spočítáme čísla \(x\), kde \(x \ge qy \land x \le qy + y - 1\), a tedy \(x \div y = q\). Podíl vynásobíme jeho počtem výskytů a tento součin sečteme napříč všemi děliteli a podíly. Vyjde nám součet všech podílů dvojic, který po nás zadání chce.

Tento algoritmus můžeme implementovat takto:

\begin{minted}{c}
int sum_quotients(int *array, int size, int max) {
    int sum = 0;

    for (int i = 0; i < size; i++) {
        int divisor = array[i];

        for (int quotient = 0; quotient * divisor <= max; quotient++) {
            int occurrences = 0;

            int start = quotient * divisor;
            int end = start + divisor - 1;
        
            for (int j = 0; j < size; j++) {
                if (array[j] >= start && array[j] <= end)
                    occurrences++;
            }

            sum += occurrences * quotient;
        }
    }

    return sum;
}
\end{minted}

\section{Počítáme rychleji}

Algoritmus z minulé sekce tráví hodně času počítáním čísel v seznamu. Tuto část můžeme snadno zrychlit.

Ze vstupního seznamu vytvoříme histogram, tj. pro každé číslo předem spočítáme, kolikrát se v seznamu vyskytuje.

Histogram můžeme v \(O\left(n\right)\) spočítat tímto jednoduchým algoritmem:

\begin{minted}{c}
void compute_histogram(int *dest, int *array, int size, int max) {
    for (int i = 0; i <= max; i++) {
        dest[i] = 0;
    }

    for (int i = 0; i < size; i++) {
        dest[array[i]]++;
    }
}
\end{minted}

Tuto funkci využijeme k zrychlení algoritmu z minulé sekce:

\begin{minted}{c}
int sum_quotients(int *array, int size, int max) {
    int histogram[max + 1];
    compute_histogram(histogram, array, size, max);
    
    int sum = 0;

    for (int i = 0; i < size; i++) {
        int divisor = array[i];

        for (int quotient = 0; quotient * divisor <= max; quotient++) {
            int occurrences = 0;

            int start = quotient * divisor;
            int end = start + divisor - 1;

            if (end > max) end = max;
        
            for (int j = start; j <= end; j++) {
                occurrences += histogram[j];
            }

            sum += occurrences * quotient;
        }
    }

    return sum;
}
\end{minted}

\section{Zrychlujeme znovu}

Sčítání podposloupností pole, které je v našem algoritmu prováděno na histogramu, je možné urychlit prefixovými součty. Ty snadno vypočítáme v \(O\left(n\right)\) touto funkcí:

\begin{minted}{c}
void compute_prefix_sums(int *dest, int *array, int size) {
    int sum = 0;

    for (int i = 0; i < size; i++) {
        sum += array[i];
        dest[i] = sum;
    }
}
\end{minted}

Tyto prefixové součty pak můžeme využít v našem alogritmu:

\begin{minted}{c}
int sum_quotients(int *array, int size, int max) {
    int histogram[max + 1];
    int histogram_prefix[max + 1];

    compute_histogram(histogram, array, size, max);
    compute_prefix_sums(histogram_prefix, histogram, max + 1);
    
    int sum = 0;

    for (int i = 0; i < size; i++) {
        int divisor = array[i];

        for (int quotient = 0; quotient * divisor <= max; quotient++) {
            int start = quotient * divisor;
            int end = start + divisor - 1;

            if (end > max) end = max;

            int occurrences = histogram_prefix[end] - histogram_prefix[start];
            occurrences += histogram[start];

            sum += occurrences * quotient;
        }
    }

    return sum;
}
\end{minted}

\section{Zrychlujeme naposledy}

Tento algoritmus je docela rychlý, ale narazí na problém, pokud budou všechna čísla v seznamu velmi nízká. Nakonec zrychlíme náš algoritmus ještě tak, že nebudeme procházet identické dělitele vícekrát. Místo toho, abychom procházeli původní seznam, tak budeme procházet histogram a násobit výsledky jednotlivých dělitelů jejich počtem. Hotový algoritmus tedy bude vypadat takto:

\begin{minted}{c}
int sum_quotients(int *array, int size, int max) {
    int histogram[max + 1];
    int histogram_prefix[max + 1];

    compute_histogram(histogram, array, size, max);
    compute_prefix_sums(histogram_prefix, histogram, max + 1);
    
    int sum = 0;

    for (int divisor = 1; divisor <= max; divisor++) {
        int divisor_count = histogram[divisor];

        for (int quotient = 0; quotient * divisor <= max; quotient++) {
            int start = quotient * divisor;
            int end = start + divisor - 1;

            if (end > max) end = max;

            int occurrences = histogram_prefix[end] - histogram_prefix[start];
            occurrences += histogram[start];

            sum += occurrences * quotient * divisor_count;
        }
    }

    return sum;
}
\end{minted}

\section{Asymptotická analýza}

Paměťová složitost tohoto algoritmu je triviálně \(O\left(n\right)\).

Pro spočítání součtu pro jednoho dělitele \(i\) potřebuje náš algoritmus projít \(O\left(\frac{n}{i}\right)\) intervalů. Protože náš algoritmus prochází všechny dělitele od 1 do \(n\), bude časová složitost algoritmu \(O\left(\sum_{i=1}^{n} \frac{n}{i}\right)\), neboli \(O\left(n \sum_{i=1}^{n} i^{-1}\right)\). \(\sum_{i=1}^{n} i^{-1}\) je prvek \(H_n\) z harmonické řady, o které je známo, že roste logaritmicky. Časová asymptotická složitost tohoto algoritmu je tedy \(O\left(n \log n\right)\).

V případě, že by maximální hodnota v poli měla vyšší řád, než je délka pole, by bylo možné nahradit histogram seřazeným polem dvojic hodnot a počtů. Na tomto poli by se využilo binárního vyhledávání. Časová složitost by pak byla \(O\left(n \log^2 n\right)\).

\end{document}
