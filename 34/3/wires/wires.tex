\documentclass{article}

\usepackage[a4paper, margin=2cm]{geometry}

\usepackage[utf8]{inputenc}
\usepackage[czech]{babel}
\usepackage{fvextra}
\usepackage{csquotes}
\usepackage{expl3}

\usepackage{parskip}
\usepackage[hidelinks, unicode, pdfusetitle]{hyperref}

\usepackage{float}
\usepackage{graphicx}

\title{34-3-X1 Dráteníci}
\author{Benjamin Swart}

\begin{document}

\maketitle

Předpokládejme že stačí jedna barva. Můžeme snadno vyzkoušet pro každý kabel všechny možné kombinace barev kabelů, které v něm mohou být, a zkontrolovat, zda se jedná o platné řešení problému. Pokud neúspěšně vyzkoušíme všechny, tak přejdeme na všechna řešení s dvěma barvami.

Toto řešení poběží v \(O(n^n)\).

\end{document}