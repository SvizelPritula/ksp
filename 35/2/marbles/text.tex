\documentclass{article}

\usepackage[a4paper, margin=2cm]{geometry}
\usepackage[utf8]{inputenc}
\usepackage[T1]{fontenc}

\usepackage[czech]{babel}
\usepackage{fvextra}
\usepackage{csquotes}
\usepackage{parskip}

\usepackage{float}
\usepackage{amsmath}
\usepackage{siunitx}
\usepackage[svgnames]{xcolor}
\usepackage{tikz}

\usepackage[hidelinks, unicode, pdfusetitle]{hyperref}

\newcommand{\brackets}[4]{{\color{#1}#2}#4{\color{#1}#3}}
\newcommand{\bred}[1]{\brackets{Maroon}{(}{)}{#1}}
\newcommand{\bgreen}[1]{\brackets{DarkGreen}{<}{>}{#1}}
\newcommand{\bblue}[1]{\brackets{MidnightBlue}{[}{]}{#1}}

\newcommand{\bredl}{{\color{Maroon}(}}
\newcommand{\bredr}{{\color{Maroon})}}

\newcommand{\mred}{\textcolor{Maroon}{R}}
\newcommand{\mgreen}{\textcolor{DarkGreen}{G}}
\newcommand{\mblue}{\textcolor{MidnightBlue}{B}}

\title{35-2-4 Kamínky}
\author{Benjamin Swart}

\pagenumbering{gobble}

\begin{document}
\maketitle

\section{Závorky}
\label{section:parens}

Výherní strategii pro tuto hru můžeme zapsat pomocí závorek. Každou dvojici kamínků nějaké barvy, kterou máme sebrat v rámci jednoho tahu, zapíšeme jako pár závorek dané barvy. Pro přehlednost přidělím barvám i různé tvary závorek. Například jedno z řešení řady \texttt{\mred\mblue\mblue\mgreen\mgreen\mgreen\mgreen\mred\mred\mgreen\mgreen\mred} je \texttt{\bred{\bblue{}\bgreen{\bgreen{}}}\bred{\bgreen{}}}. Takto zapsané řešení můžeme snadno zahrát, a to tak, že pro sebrání závorky nejprve sebereme všechny závorky v ní, a poté sebereme vnější závorku.

\section{Optimální hra}

V této hře není možné hrát suboptimálně. Každou hru buď nelze vyhrát, nebo nelze prohrát.

Dokážeme to sporem. Předpokládejme, že existuje nějaká řada kamínků, výherní strategie pro tuto řadu a platný tah, po kterém zbude neřešitelná řada. Výherní strategii můžeme zapsat pomocí závorek, jak je popsáno v sekci \ref{section:parens}. Náš tah z popisu strategie odebírá dvě sousedící závorky stejné barvy. Mohou nastat čtyři situace:

\subsection[Tah odebírá ()]{Tah odebírá \texttt{\bred{}}}

Tento tah odpovídá strategii, takže po jeho zahrání určitě můžeme stále vyhrát.

\subsection[Tah odebírá )(]{Tah odebírá \texttt{\bredr\bredl}}

Naši řadu kamínků můžeme zapsat takto:

\begin{center}
    \texttt{{A}\bredl{B}\underline{\bredr\bredl}{C}\bredr{D}}
\end{center}

Odebíráme závorky jsou podtržené, písmena označují řešitelné podsekvence. Po odebrání závorek nám zbude takováto řada:

\begin{center}
    \texttt{{A}\bredl{B}{C}\bredr{D}}
\end{center}

Tato řada je zjevně také řešitelná.

\subsection[Tah odebírá ((]{Tah odebírá \texttt{\bredl\bredl}}

Naši řadu kamínků můžeme zapsat takto:

\begin{center}
    \texttt{{A}\underline{\bredl\bredl}{B}\bredr{C}\bredr{D}}
\end{center}

Po odebrání nám zbude \texttt{{A}{B}\bredr{C}\bredr{D}}. První závorku otočíme a dostaneme tuto řadu:

\begin{center}
    \texttt{{A}{B}\bredl{C}\bredr{D}}
\end{center}

Tato řada je také zjevně řešitelná.

\subsection[Tah odebírá ))]{Tah odebírá \texttt{\bredr\bredr}}

Máme tuto řadu:

\begin{center}
    \texttt{{A}\bredl{B}\bredl{C}\underline{\bredr\bredr}{D}}
\end{center}

Tu obdobně jako \texttt{\bredl\bredl} vyřešíme takto:

\begin{center}
    \texttt{{A}\bredl{B}\bredr{C}{D}}
\end{center}

Ať tedy zahrajeme jakýkoliv tah, tak jsme schopni upravit naši strategii, abychom vyhráli.

\section{Simulace}

Stačí tedy dělat náhodné tahy, dokud nedojdou. Pokud tím odstraníme všechny kamínky, tak jsme vyhráli. Pokud nějaké zbudou, tak vyhrát možné není. Stačí tedy hru efektivně simulovat.

Barevné kamínky si uložíme do spojového seznamu. Na začátku ho projdeme a najdeme v něm všechny kamínky, které jsou následovány kamínkem stejné barvy. Odkazy na ně dáme do fronty. Tato fronta bude obsahovat všechny aktuálně možné tahy.

Vždy z fronty jeden tah odebereme a zahrajeme ho tak, že ze seznamu odstraníme daný kamínek dohromady s kamínkem po něm následujícím. Poté zkontrolujeme, jestli jsme nespojili dva kamínky stejné barvy. Pokud ano, tak první z nich přidáme do fronty.

Může se stát, že zahráním tahu nějakou dvojici narušíme, a tedy ve frontě skončí neplatný tah. Musíme si tedy dát pozor, abychom neplatné tahy ignorovali a nehráli. Každopádně vždy ve frontě budou všechny platné tahy.

Do fronty na začátku vložíme nejvýše $n - 1$ prvků. Další prvek do ní vložíme jen tehdy, když zahrajeme platný tah, a to pro každý tak nejvýše jeden. Tahů za hru zahrajeme nejvýše $\frac{n}{2}$. Do fronty tedy celkem vložíme $\mathcal{O}\left(n\right)$ prvků, a stejně jich vyndáme. Asymptotická časová i paměťová složitost tohoto algoritmu je tedy $\mathcal{O}\left(n\right)$.

\end{document}
