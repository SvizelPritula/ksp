\documentclass{article}

\usepackage[a4paper, margin=2cm]{geometry}
\usepackage[utf8]{inputenc}
\usepackage[T1]{fontenc}

\usepackage[czech]{babel}
\usepackage{fvextra}
\usepackage{csquotes}
\usepackage{parskip}

\usepackage{float}
\usepackage{amsmath}
\usepackage{siunitx}
\usepackage{tikz}

\usepackage[hidelinks, unicode, pdfusetitle]{hyperref}

\title{35-2-2 Zápisky z přednášky}
\author{Benjamin Swart}

\begin{document}
\maketitle

$k$-té nejmenší číslo v haldě $H$ bude nutně dítětem jednoho z prvních $k-1$ nejmenších čísel. Stačí tedy najít $k-1$ nejmenších čísel a projít jejich děti. $k$ nejmenších čísel tedy můžeme najít dynamickým programováním. Z haldy $H$ budeme postupně škrtat\footnote{Algoritmus na odebírání z haldy pochopitelně používat nebudeme. Škrtání bude pouze pomyslné.} čísla a udržovat si kolekci neprozkoumaných dětí.

Za tímto účelem si vytvoříme haldu $S$, do které přidáme odkaz na kořen haldy $H$. Z $S$ postupně odebereme $k$ čísel. Pokaždé, když z $S$ odebereme nějaké číslo, tak do ní přidáme oba potomky daného čísla z $H$. $k$-té odebrané číslo vrátíme.

Takto může vypadat průběh algoritmu pro $k = 5$:

\begin{center}
    \begin{tikzpicture}
        \tikzstyle{level 1}=[sibling distance=20mm]
        \tikzstyle{level 2}=[sibling distance=10mm]
        \tikzstyle{every node}=[circle, draw]

        \node[ultra thick]{$1$}
        child{
                node{$4$}
                child{
                        node{$8$}
                    }
                child{
                        node{$7$}
                    }
            }
        child{
                node{$2$}
                child{
                        node{$3$}
                    }
                child{
                        node{$5$}
                    }
            };
    \end{tikzpicture}
    \begin{tikzpicture}
        \tikzstyle{level 1}=[sibling distance=20mm]
        \tikzstyle{level 2}=[sibling distance=10mm]
        \tikzstyle{every node}=[circle, draw]

        \node[thick, fill=lightgray]{$1$}
        child{
                node[ultra thick]{$4$}
                child{
                        node{$8$}
                    }
                child{
                        node{$7$}
                    }
            }
        child{
                node[ultra thick]{$2$}
                child{
                        node{$3$}
                    }
                child{
                        node{$5$}
                    }
            };
    \end{tikzpicture}
    \begin{tikzpicture}
        \tikzstyle{level 1}=[sibling distance=20mm]
        \tikzstyle{level 2}=[sibling distance=10mm]
        \tikzstyle{every node}=[circle, draw]

        \node[thick, fill=lightgray]{$1$}
        child{
                node[ultra thick]{$4$}
                child{
                        node{$8$}
                    }
                child{
                        node{$7$}
                    }
            }
        child{
                node[thick, fill=lightgray]{$2$}
                child{
                        node[ultra thick]{$3$}
                    }
                child{
                        node[ultra thick]{$5$}
                    }
            };
    \end{tikzpicture}

    \begin{tikzpicture}
        \tikzstyle{level 1}=[sibling distance=20mm]
        \tikzstyle{level 2}=[sibling distance=10mm]
        \tikzstyle{every node}=[circle, draw]

        \node[thick, fill=lightgray]{$1$}
        child{
                node[ultra thick]{$4$}
                child{
                        node{$8$}
                    }
                child{
                        node{$7$}
                    }
            }
        child{
                node[thick, fill=lightgray]{$2$}
                child{
                        node[thick, fill=lightgray]{$3$}
                    }
                child{
                        node[ultra thick]{$5$}
                    }
            };
    \end{tikzpicture}
    \begin{tikzpicture}
        \tikzstyle{level 1}=[sibling distance=20mm]
        \tikzstyle{level 2}=[sibling distance=10mm]
        \tikzstyle{every node}=[circle, draw]

        \node[thick, fill=lightgray]{$1$}
        child{
                node[thick, fill=lightgray]{$4$}
                child{
                        node[ultra thick]{$8$}
                    }
                child{
                        node[ultra thick]{$7$}
                    }
            }
        child{
                node[thick, fill=lightgray]{$2$}
                child{
                        node[thick, fill=lightgray]{$3$}
                    }
                child{
                        node[ultra thick]{$5$}
                    }
            };
    \end{tikzpicture}
    \begin{tikzpicture}
        \tikzstyle{level 1}=[sibling distance=20mm]
        \tikzstyle{level 2}=[sibling distance=10mm]
        \tikzstyle{every node}=[circle, draw]

        \node[thick, fill=lightgray]{$1$}
        child{
                node[thick, fill=lightgray]{$4$}
                child{
                        node[ultra thick]{$8$}
                    }
                child{
                        node[ultra thick]{$7$}
                    }
            }
        child{
                node[thick, fill=lightgray]{$2$}
                child{
                        node[thick, fill=lightgray]{$3$}
                    }
                child{
                        node[thick, fill=lightgray]{$5$}
                    }
            };
    \end{tikzpicture}
\end{center}

Vrcholy v $S$ jsou označeny tučnou čarou, vrcholy co byly z $S$ odebrány jsou označeny šedě.

Časová složitost tohoto algoritmu je $\mathcal{O}\left(k \log{k}\right)$. Kromě paměti potřebné pro uložení $H$ potřebuje $\mathcal{O}\left(k\right)$ další paměti.

\end{document}
