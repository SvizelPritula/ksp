\documentclass{article}

\usepackage[a4paper, margin=2cm]{geometry}
\usepackage[utf8]{inputenc}
\usepackage[T1]{fontenc}

\usepackage[czech]{babel}
\usepackage{fvextra}
\usepackage{csquotes}
\usepackage{parskip}

\usepackage{float}
\usepackage{amsmath}
\usepackage{siunitx}
\usepackage{graphicx}

\usepackage[hidelinks, unicode, pdfusetitle]{hyperref}

\graphicspath{{images}}

\title{35-5-4 Kalkulačka}
\author{Benjamin Swart}

\begin{document}
\maketitle

\section{Univerzální suboptimální řešení}

Nejprve pár poznatků. Každé číslo je možné vyrobit tak, že jej převedeme do devítkové soustavy. Do kalkulačky nejprve vložíme první cifru tohoto čísla, poté projdeme zbylé cifry a každou přičteme poté, co mezivýsledek vynásobíme devíti. Pokud je číslo záporné, tak cifry odčítáme a první cifru nejprve odečteme od nuly. Například číslo 25 můžeme vyrobit příkladem \texttt{2*9+7}. Samozřejmě ho však můžeme vyrobit i jako \texttt{5*5}, takže je naše řešení suboptimální. Tato metoda nám však stanoví horní mez pro počet operací na $2 * \lceil\log_9 K\rceil - 1$ pro kladná $K$ a $2 * \lceil\log_9 \left(-K\right)\rceil$ pro záporná $K$.

\section{Potřebné operace}
\label{needed}

Operace \textit{modulo} je zjevně zbytečná. Nemůžeme s ní vyrobit žádná čísla která nemůžeme získat s nulovou spotřebou, protože $n \mod m \in \left<0; m\right>$ a $m \in \left<0; 9\right>$.

Odčítání je nezbytně nutné k vytvoření záporných čísel, jakožto jediná operace, která nezachovává vždy znaménko. Jediný způsob, jak získat záporné číslo, je odečtením $m \in \left<1; 9\right>$ od $m \in \left<0; m - 1\right>$. Naopak jediný způsob jak získat kladné číslo je obdobně přičtení cifry k malému zápornému číslu.

Pokud se nám změní znaménko čísla na displeji, bude nutně nové číslo $n$ splňovat $\lvert n \rvert \leq 9$. Taková čísla dokážeme vyrobit snadno, takže nedává smysl během nastavování měnit znaménko. Pokud chceme vytvořit kladné číslo, tak nemá smysl mít mezivýsledek záporný, a pokud vyrábíme záporné číslo, tak je nejlepší volbou na začátku odečíst nějakou cifru od nuly a poté mít mezivýsledek záporný vždy.

Bohužel neplatí, že by nikdy nedávalo smysl použít dělení. Například číslo $853$ se dá nejlépe vytvořit pomocí příkladu \texttt{5*8*8*8/3}. Neexistuje stejně dlouhá posloupnost operací bez dělení, která by číslo $853$ vytvořila. Využívá se zde toho, že dělení zaokrouhluje dolů, jinak by bylo dělení opravdu zbytečné. Optimální příklad pro výrobu čísla $1339231$ dokonce používá dělení dvě: \texttt{7*9*9*9*9*9*9*9/5/5}.

\section{Algoritmus}

Budeme prohledávat stavový prostor prohledáváním do šířky. Začneme s čísly $0, \dots, 9$ a skončíme, jakmile narazíme na hledané číslo $K$. Můžeme vyhledávání omezit jen na tu polovinu číselné osy, ve které se nachází $K$, jak bylo popsáno v sekci \ref{needed}.

Časová složitost prohledávání do hloubky je v případě konstantního počtu operací lineární ku počtu prohledaných stavů. Víme, že k nalezení hledaného čísla nepotřebujeme více než $2\lceil\log_9 K\rceil$ kroků. Z našich operací roste nejrychleji násobení devíti, takže největší číslo, které můžeme při vyhledávání potkat, bude $9^{2\lceil\log_9 K\rceil}$. Toto číslo bude určitě menší než $9^{2 \left(1 + \log_9 K\right)} = 81 K^2$, takže časová i paměťová složitost bude nejhůře $\mathcal{O}\left(\lvert K \rvert^2\right)$.

\end{document}
