\documentclass{article}

\usepackage[a4paper, margin=2cm]{geometry}
\usepackage[utf8]{inputenc}

\usepackage[czech]{babel}
\usepackage{fvextra}
\usepackage{csquotes}
\usepackage{parskip}

\usepackage{float}
\usepackage{amsmath}
\usepackage{siunitx}

\usepackage[hidelinks, unicode, pdfusetitle]{hyperref}

\title{Inverzní matice}
\author{Benjamin Swart}

\pagenumbering{gobble}

\begin{document}

Máme jednoduchou matici $C$, kterou chceme invertovat:

\begin{equation*}
    \begin{pmatrix}
        c & -c \\
        c & c
    \end{pmatrix}
\end{equation*}

Rovnici $C v = v^\prime$ si můžeme zapsat jako soustavu rovnic.

\begin{align*}
    \begin{pmatrix}
        c & -c \\
        c & c
    \end{pmatrix}
    \begin{pmatrix}
        v_1 \\
        v_2
    \end{pmatrix}
                  & =
    \begin{pmatrix}
        v^\prime_1 \\
        v^\prime_2
    \end{pmatrix}               \\
    \begin{pmatrix}
        c v_1 - c v_2 \\
        c v_1 + c v_2
    \end{pmatrix}
                  & =
    \begin{pmatrix}
        v^\prime_1 \\
        v^\prime_2
    \end{pmatrix}               \\
    \\
    c v_1 - c v_2 & = v^\prime_1 \\
    c v_1 + c v_2 & = v^\prime_2
\end{align*}

Z této soustavy můžeme vytknout $v_1$ a $v_2$.

\begin{align*}
    c v_1 - c v_2   & = v^\prime_1                        \\
    c v_1 + c v_2   & = v^\prime_2                        \\
    \\
    2 c v_1         & = v^\prime_1 + v^\prime_2           \\
    v_1             & =
    \tfrac{1}{2} c^{-1} v^\prime_1 +
    \tfrac{1}{2} c^{-1} v^\prime_2                        \\
    \\
    c \left(
    \tfrac{1}{2} c^{-1} v^\prime_1 +
    \tfrac{1}{2} c^{-1} v^\prime_2
    \right) + c v_2 & = v^\prime_2                        \\
    \tfrac{1}{2} v^\prime_1 + \tfrac{1}{2} v^\prime_2
    + c v_2         & = v^\prime_2                        \\
    c v_2           & = v^\prime_2 -
    \tfrac{1}{2} v^\prime_1 - \tfrac{1}{2} v^\prime_2     \\
    c v_2           & = \tfrac{1}{2} c^{-1} v^\prime_2 -
    \tfrac{1}{2} v^\prime_1                               \\
    v_2             & = -\tfrac{1}{2} c^{-1} v^\prime_1 +
    \tfrac{1}{2} c^{-1} v^\prime_2                        \\
    \\
    \begin{pmatrix}
        \tfrac{1}{2} c^{-1} v^\prime_1 +
        \tfrac{1}{2} c^{-1} v^\prime_2 \\
        - \tfrac{1}{2} c^{-1} v^\prime_1 +
        \tfrac{1}{2} c^{-1} v^\prime_2
    \end{pmatrix}
                    & =
    \begin{pmatrix}
        v_1 \\
        v_2
    \end{pmatrix}                                        \\
\end{align*}

Tento výraz můžeme převést do podoby násobení matice.

\begin{align*}
    \begin{pmatrix}
        \tfrac{1}{2} c^{-1}   & \tfrac{1}{2} c^{-1} \\
        - \tfrac{1}{2} c^{-1} & \tfrac{1}{2} c^{-1}
    \end{pmatrix}
    \begin{pmatrix}
        v^\prime_1 \\
        v^\prime_2
    \end{pmatrix}
     & =
    \begin{pmatrix}
        v_1 \\
        v_2
    \end{pmatrix}
\end{align*}

Tím získáme inverzní matici $C^{-1}$:

\begin{equation*}
    \begin{pmatrix}
        \tfrac{1}{2} c^{-1}   & \tfrac{1}{2} c^{-1} \\
        - \tfrac{1}{2} c^{-1} & \tfrac{1}{2} c^{-1}
    \end{pmatrix}
\end{equation*}

Můžeme ověřit, že $C C^{-1} = E$.

\begin{align*}
    \begin{pmatrix}
        c & -c \\
        c & c
    \end{pmatrix}
    \begin{pmatrix}
        \tfrac{1}{2} c^{-1}   & \tfrac{1}{2} c^{-1} \\
        - \tfrac{1}{2} c^{-1} & \tfrac{1}{2} c^{-1}
    \end{pmatrix}
     & =
    \begin{pmatrix}
        c \left(\tfrac{1}{2} c^{-1}\right)
        - c \left(- \tfrac{1}{2} c^{-1}\right) &
        c \left(\tfrac{1}{2} c^{-1}\right)
        - c \left(\tfrac{1}{2} c^{-1}\right)
        \\
        c \left(\tfrac{1}{2} c^{-1}\right)
        + c \left(-\tfrac{1}{2} c^{-1}\right)  &
        c \left(\tfrac{1}{2} c^{-1}\right)
        + c \left(\tfrac{1}{2} c^{-1}\right)
    \end{pmatrix} \\
     & =
    \begin{pmatrix}
        \tfrac{1}{2} + \tfrac{1}{2} &
        \tfrac{1}{2} - \tfrac{1}{2}
        \\
        \tfrac{1}{2} - \tfrac{1}{2} &
        \tfrac{1}{2} + \tfrac{1}{2}
    \end{pmatrix}            \\
     & =
    \begin{pmatrix}
        1 & 0 \\
        0 & 1
    \end{pmatrix}
\end{align*}

\end{document}
