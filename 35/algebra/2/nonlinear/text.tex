\documentclass{article}

\usepackage[a4paper, margin=2cm]{geometry}
\usepackage[utf8]{inputenc}

\usepackage[czech]{babel}
\usepackage{fvextra}
\usepackage{csquotes}
\usepackage{parskip}

\usepackage{float}
\usepackage{amsmath}
\usepackage{siunitx}

\usepackage[hidelinks, unicode, pdfusetitle]{hyperref}

\title{Nelineární zobrazení}
\author{Benjamin Swart}

\pagenumbering{gobble}

\begin{document}

Příkladem nelineárního zobrazení je translace, například $f\left(v\right) = v + \begin{pmatrix}
        1 \\
        0
    \end{pmatrix}$.

Toto zobrazení přesune bod doprava o jednotkovou délku. Toto zobrazení lineární není, jelikož není distributivní, a to dokonce pro žádnou dvojici vektorů. Pokud máme vektory $x$ a $y$, tak bude $f\left(x + y\right) = x + y + \begin{pmatrix} 1 \\ 0 \end{pmatrix}$, zatímco $f\left(x\right) + f\left(y\right) = x + y + \begin{pmatrix} 2 \\ 0 \end{pmatrix}$.

\end{document}
