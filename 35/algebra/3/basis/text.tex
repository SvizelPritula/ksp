\documentclass{article}

\usepackage[a4paper, margin=2cm]{geometry}
\usepackage[utf8]{inputenc}

\usepackage[czech]{babel}
\usepackage{fvextra}
\usepackage{csquotes}
\usepackage{parskip}

\usepackage{float}
\usepackage{amsmath}
\usepackage{siunitx}
\usepackage{tensor}
\usepackage{icomma}

\usepackage[hidelinks, unicode, pdfusetitle]{hyperref}

\title{Matice dosažitelnosti}
\author{Benjamin Swart}

\pagenumbering{gobble}

\begin{document}

Chceme najít matice převodu mezi následujícími bázemi prostoru kvadratických funkcí:

\begin{itemize}
    \item $B$ - Komponenty reprezentují koeficienty funkce: $
              \begin{bmatrix}
                  c \\
                  b \\
                  a
              \end{bmatrix}
          $
    \item $C$ - Komponenty reprezentují hodnoty funkce pro $x = -1; 0; 1$: $
              \begin{bmatrix}
                  q(-1) \\
                  q(0)  \\
                  q(1)
              \end{bmatrix}
              =
              \begin{bmatrix}
                  a - b + c \\
                  c         \\
                  a + b + c
              \end{bmatrix}
          $
\end{itemize}

Matici přechodu z $B$ do $C$ snadno získáme tak, že vyhodnotíme funkce $1$, $x$ a $x^2$ pro $x = -1; 0; 1$.

\begin{align*}
    \tensor[_C]{[id]}{_B} & = \begin{bmatrix}
                                  1 & -1 & 1 \\
                                  1 & 0  & 0 \\
                                  1 & 1  & 1
                              \end{bmatrix}
\end{align*}

Pro sestavení matice přechodu z $C$ do $B$ musíme najít kvadratické funkce, které mají v bodech $x = -1; 0; 1$ správné hodnoty. Jinak řečeno hledáme vektory, kterými musíme vynásobit $\tensor[_C]{[id]}{_B}$, abychom získali jednotkové vektory v bázi $C$. Ty snadno spočítáme Gaussovou eliminační metodou.\footnote{Můžeme využít program z první série.}

\begin{align*}
    \begin{bmatrix}
        1 & -1 & 1 \\
        1 & 0  & 0 \\
        1 & 1  & 1 \\
    \end{bmatrix} [c_1]_B & = \begin{bmatrix}
                                  1 \\
                                  0 \\
                                  0 \\
                              \end{bmatrix} \\
    [c_1]_B               & = \begin{bmatrix}
                                  0    \\
                                  -0,5 \\
                                  0,5  \\
                              \end{bmatrix} \\\\
    \begin{bmatrix}
        1 & -1 & 1 \\
        1 & 0  & 0 \\
        1 & 1  & 1 \\
    \end{bmatrix} [c_2]_B & = \begin{bmatrix}
                                  0 \\
                                  1 \\
                                  0 \\
                              \end{bmatrix} \\
    [c_2]_B               & = \begin{bmatrix}
                                  1  \\
                                  0  \\
                                  -1 \\
                              \end{bmatrix} \\\\
    \begin{bmatrix}
        1 & -1 & 1 \\
        1 & 0  & 0 \\
        1 & 1  & 1 \\
    \end{bmatrix} [c_3]_B & = \begin{bmatrix}
                                  0 \\
                                  0 \\
                                  1 \\
                              \end{bmatrix} \\
    [c_3]_B               & = \begin{bmatrix}
                                  0   \\
                                  0,5 \\
                                  0,5 \\
                              \end{bmatrix}
\end{align*}

Matice přechodu z $C$ do $B$ tedy bude vypadat takto:

\begin{align*}
    \tensor[_B]{[id]}{_C} & = \begin{bmatrix}
                                  0    & 1  & 0   \\
                                  -0,5 & 0  & 0,5 \\
                                  0,5  & -1 & 0,5
                              \end{bmatrix}
\end{align*}

Zobrazení $\tensor[_C]{[g]}{_C}$ do báze $B$ tedy převedeme takto:

\begin{align*}
    \tensor[_B]{[g]}{_B} & = \tensor[_B]{[id]}{_C} * \tensor[_C]{[g]}{_C} * \tensor[_C]{[id]}{_B} \\
    \tensor[_B]{[g]}{_B} & =
    \begin{bmatrix}
        0    & 1  & 0   \\
        -0,5 & 0  & 0,5 \\
        0,5  & -1 & 0,5
    \end{bmatrix}
    * \begin{bmatrix}
          0 & 0 & 1 \\
          0 & 1 & 0 \\
          1 & 0 & 0 \\
      \end{bmatrix}
    * \begin{bmatrix}
          1 & -1 & 1 \\
          1 & 0  & 0 \\
          1 & 1  & 1
      \end{bmatrix}                                                                              \\
    \tensor[_B]{[g]}{_B} & =
    \begin{bmatrix}
        1 & 0  & 0 \\
        0 & -1 & 0 \\
        0 & 0  & 1 \\
    \end{bmatrix}                                                                                \\
\end{align*}

Tato matice se dá snadno odvodit. Zrcadlení funkce znamená, že invertujeme parametr $x$, čímž z funkce $f(x) = ax^2 + bx + c$ získáme $f_z(x) = ax^2 - bx + c$. Také na tuto operaci můžeme pohlížet tak, že zrcadlíme jednotlivé souřadnice báze. $x^2$ i $1$ jsou sudé funkce, zatímco $x$ je funkce lichá. $x^2$ a $1$ tedy necháme beze změny a u $x$ změníme znaménko.

\end{document}
