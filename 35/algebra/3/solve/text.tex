\documentclass{article}

\usepackage[a4paper, margin=2cm]{geometry}
\usepackage[utf8]{inputenc}

\usepackage[czech]{babel}
\usepackage{fvextra}
\usepackage{csquotes}
\usepackage{parskip}

\usepackage{float}
\usepackage{amsmath}
\usepackage{siunitx}
\usepackage{tensor}
\usepackage{icomma}

\usepackage[hidelinks, unicode, pdfusetitle]{hyperref}

\title{Matice dosažitelnosti}
\author{Benjamin Swart}

\pagenumbering{gobble}
\setcounter{secnumdepth}{0}

\begin{document}

\section{Určení počtu řešení}

Nejprve zjistíme, jestli má tato rovnice vůbec nějaké řešení, neboli jestli $b \in S\left(A\right)$. To zjistíme tak, že ověříme, jestli jsou vektory $\left\{b, s_1, s_2, \dots,s_r\right\}$ lineárně závislé. To můžeme udělat Gaussovou eliminací.

Tato metoda má časovou složitost $\mathcal{O}\left(m r^2\right)$, zatímco přímočaré řešení má složitost $\mathcal{O}\left(m n \min\left(m, n\right)\right)$. Bude tedy efektivnější pro matice s velkým $k$.

\section{Vyjádření množiny řešení}

Jakmile máme jedno řešení $x_0$, tak můžeme množinu všech řešení snadno zapsat ve tvaru $x_0 + x_K; x_K \in \ker\left(A\right)$.

Tato metoda je v porovnání s Gaussovou eliminací nesrovnatelně rychlejší, jelikož nezabere téměř žádný čas. Přesná časová složitost závisí na požadovaném formátu řešení, ale v podobě součtu vektorů s proměnnými koeficienty jej dokážeme vypsat v $\mathcal{O}\left(nk\right)$.

\end{document}
