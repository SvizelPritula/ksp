\documentclass{article}

\usepackage[a4paper, margin=2cm]{geometry}
\usepackage[utf8]{inputenc}
\usepackage[T1]{fontenc}

\usepackage[czech]{babel}
\usepackage{fvextra}
\usepackage{csquotes}
\usepackage{parskip}

\usepackage{float}
\usepackage{amsmath}
\usepackage{siunitx}
\usepackage{graphicx}

\usepackage[hidelinks, unicode, pdfusetitle]{hyperref}

\graphicspath{{images}}

\title{Mnohoúhelník}
\author{Benjamin Swart}

\begin{document}

Máme bod $x$ a rovinu $\rho$:

\begin{align*}
    x =   & \begin{bmatrix}
                6 \\ 3 \\ 3
            \end{bmatrix} \\
    \rho: & \ c + U a      \\
    =     &
    \begin{bmatrix}
        1 \\ 2 \\ 3
    \end{bmatrix} +
    \begin{bmatrix}
        2  & 4 \\
        -3 & 4 \\
        6  & 2
    \end{bmatrix}
    \begin{bmatrix}
        a_1 \\
        a_2
    \end{bmatrix}
\end{align*}

Vzdálenost se určitě nezmění, pokud oba útvary posuneme. Od obou odečteme vektor $c$. (Pro posunuté útvary použiji původní jména, je zbytečné přiřazovat nová.)

\begin{align*}
    x =   & \begin{bmatrix}
                5 \\
                1 \\
                0
            \end{bmatrix} \\
    \rho: & \ U a          \\
    =     &
    \begin{bmatrix}
        2  & 4 \\
        -3 & 4 \\
        6  & 2
    \end{bmatrix}
    \begin{bmatrix}
        a_1 \\
        a_2
    \end{bmatrix}
\end{align*}

Nyní prochází $\rho$ počátkem soustavy souřadnic. Abychom určili vzdálenost bodu $x$ od roviny $\rho$, tak najdeme normalizovaný vektor kolmý na $\rho$ a pak určíme velikost projekce $x$ na $\rho$.

Vektor kolmý na $\rho$ najdeme pomocí jádra matice. Jádro matice je množinou všech vektorů kolmých na řádky matice, takže spočítáme jádro matice, které má báze naší roviny jako řádky, neboli $U^T$. Pomocí kalkulačky z prvního dílu můžeme vypočítat, že řešením jsou násobky vektoru $\left[-3, 2, 2\right]^T$, který můžeme normalizovat:

\begin{align*}
    n & = \sqrt{17}^{-1}
    \begin{bmatrix}
        -3 \\ 2 \\ 2
    \end{bmatrix}                         \\
      & = \begin{bmatrix}
              \frac{-3}{\sqrt{17}} \\
              \frac{2}{\sqrt{17}}  \\
              \frac{2}{\sqrt{17}}
          \end{bmatrix}
\end{align*}

Nakonec spočítáme projekci skalárním součinem:

\begin{align*}
    n \cdot x & =
    \begin{bmatrix}
        \frac{-3}{\sqrt{17}} \\
        \frac{2}{\sqrt{17}}  \\
        \frac{2}{\sqrt{17}}
    \end{bmatrix}
    \cdot
    \begin{bmatrix}
        5 \\ 1 \\ 0
    \end{bmatrix}
    = -\frac{13}{\sqrt{17}}
\end{align*}

Vzdálenost bude absolutní hodnota výsledku, neboli $\frac{13}{\sqrt{17}}$.

\end{document}
