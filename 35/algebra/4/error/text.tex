\documentclass{article}

\usepackage[a4paper, margin=2cm]{geometry}
\usepackage[utf8]{inputenc}
\usepackage[T1]{fontenc}

\usepackage[czech]{babel}
\usepackage{fvextra}
\usepackage{csquotes}
\usepackage{parskip}

\usepackage{float}
\usepackage{amsmath}
\usepackage{siunitx}
\usepackage{graphicx}

\usepackage[hidelinks, unicode, pdfusetitle]{hyperref}

\graphicspath{{images}}

\title{Mnohoúhelník}
\author{Benjamin Swart}

\begin{document}

Lineární funkce jsou funkce ve tvaru $f(x) = ax + b$. Můžeme je tedy reprezentovat jako vektor ve tvaru $\left[a, b\right]^T$. Pokud spočítáme skalární součin tohoto vektoru s vektorem $\left[x, 1\right]^T$, tak nám vyjde hodnota $f(x)$.

Násobení matice vektorem spočítá vektor složený ze skalárních součinů řádků matice s daným vektorem. Požadavky na hledanou funkci tedy můžeme vyjádřit touto rovnicí:

\begin{align*}
    A f & = b \\
    \begin{bmatrix}
        0  & 1 \\
        3  & 1 \\
        10 & 1 \\
        14 & 1 \\
        20 & 1 \\
    \end{bmatrix} f
        & =
    \begin{bmatrix}
        67 \\
        66 \\
        63 \\
        62 \\
        60 \\
    \end{bmatrix}
\end{align*}

Tato soustava řešení nemá, ale můžeme použít metodu nejmenších čtverců k nalezení přibližného řešení. Budeme tím minimalizovat čtverce rozdílu mezi chtěnými a skutečnými hodnotami funkce v daných bodech, což je přesně to, co chceme. Postupujeme tedy dle vzorců ze studijního textu:

\begin{align*}
    A^T A f & = A^Tb \\
    \begin{bmatrix}
        0 & 3 & 10 & 14 & 20 \\
        1 & 1 & 1  & 1  & 1  \\
    \end{bmatrix}
    \begin{bmatrix}
        0  & 1 \\
        3  & 1 \\
        10 & 1 \\
        14 & 1 \\
        20 & 1 \\
    \end{bmatrix}
    f       & =
    \begin{bmatrix}
        0 & 3 & 10 & 14 & 20 \\
        1 & 1 & 1  & 1  & 1  \\
    \end{bmatrix}
    \begin{bmatrix}
        67 \\
        66 \\
        63 \\
        62 \\
        60 \\
    \end{bmatrix}   \\
    \begin{bmatrix}
        705 & 47 \\
        47  & 5  \\
    \end{bmatrix} f
            & =
    \begin{bmatrix}
        2896 \\
        318
    \end{bmatrix}   \\
    f       & =
    \begin{bmatrix}
        -\frac{233}{658} \\
        \frac{937}{14}
    \end{bmatrix}
\end{align*}

Získáváme tedy tuto funkci:

\begin{align*}
    f(x) & = -\frac{233}{658} x + \frac{937}{14}
\end{align*}

\end{document}
