\documentclass{article}

\usepackage[a4paper, margin=2cm]{geometry}
\usepackage[utf8]{inputenc}
\usepackage[T1]{fontenc}

\usepackage[czech]{babel}
\usepackage{fvextra}
\usepackage{csquotes}
\usepackage{parskip}

\usepackage{float}
\usepackage{amsmath}
\usepackage{siunitx}

\usepackage[hidelinks, unicode, pdfusetitle]{hyperref}

\title{35-3-2 Hroší hortenzie}
\author{Benjamin Swart}

\begin{document}
\maketitle

Pro nalezení optimální mističky pro nějakého kamaráda musíme provést dva kroky:

\begin{enumerate}
        \item \label{step-one} Najdeme první a poslední mystičku, která produkuje čaj o přijatelné koncentraci.
        \item \label{step-two} V nalezeném intervalu mističek najdeme tu s nejnižší spotřebou listů.
\end{enumerate}

Mističky seřadíme podle koncentrace hippopodulcinu v čaji, který produkují. To nám umožní efektivně provést \hyperref[step-one]{první} krok pomocí binárního vyhledávání. Také nad seznamem misek postavíme minimový intervalový strom podle spotřeby listů, který nám umožní efektivně provést \hyperref[step-two]{druhý} krok.

Setřídit mističky podle koncentrace dokážeme v $\mathcal{O}(k\log k)$, postavit nad nimi intervalový strom trvá také $\mathcal{O}(k\log k)$. Pro každého z $n$ kamarádů dokážeme v $\mathcal{O}(\log k)$ najít interal přijatelných mističkek, a v $\mathcal{O}(\log k)$ také pomocí intervalového stromu najdeme v tomto intervalu mističku s nejnižší spotřebou.\footnote{Pomocí minimového intervalového stromu v daném intervalu snadno najdeme nejmenší hodnotu. Potřebujeme ale vědět, které mističce tato hodnota patří. Toho nejsnáze dosáhneme tak, že ve vrcholech stromu budeme kromě nejnižší hodnoty ukládat i číslo mističky, která tuto hodnotu má.} Celý algoritmus tedy poběží v $\mathcal{O}((n + k) \log k)$. Největší objekt v paměti bude intervalový strom, který potřebuje $\mathcal{O}(k\log k)$ paměti. Ke kamarádům dokážeme přiřazovat mističky onlinově, a nemusíme tedy jejich preference udržovat v paměti, pokud je můžeme zpracovávat postupně. V opačném případě bychom potřebovali $\mathcal{O}(n)$ paměti na uložení jejich preferencí.

\end{document}
